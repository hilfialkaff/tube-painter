\section{Implementation}\label{sec:impl}

Our \tube system has two components; the tube itself and the screen. As shown in Figure~\ref{fig:impl1}, the tube is attached to an acrylic enclosure that houses a small Force-Sensing Resistor (FSR) and an Inertial Measurement Unit (IMU) that possesses 5 degrees of freedom from 3-axis accelerometer and 2-axis gyroscope. The IMU captures the three-dimensional motion of the tube and translates it into 2-dimensional position on the screen using rigid body kinematics and Euler’s discretization method for the integration. Additionally, it also records the angle in which the tube is rotated and how fast it is rotating. With the FSR integrated in the tube, the magnitude of the user’s breathing into the tube is also captured. All of these informations will then be passed into Arduino which then will be read by a processing module.

\begin{figure}
  \centering
  \includegraphics[width=\linewidth]{./figs/impl1.png}
  \caption{Our \tube with all the sensors.}
  \label{fig:impl1}
\end{figure}

Figure~\ref{fig:design-sketch} shows an overview of how our \tube system works as a whole. As we move or rotate \tube, our movements are recorded by an Arduino microcontroller. Arduino will then send all of the data it received to be used by the applications in the computer.

\begin{figure}
  \centering
  \includegraphics[width=0.8\linewidth]{./figs/sketch.png}
  \caption{Overview of how our system works.}
  \label{fig:design-sketch}
\end{figure}

We have developed two applications to demonstrate the uniqueness and interactiveness of \tube. The applications that we developed are written in processing since it provides a smooth interface with Arduino while boasting numerous easy-to-use graphical functions. Making new processing applications to work with our \tube requires very minimal changes to the existing base code since we have made the interface to the hardware simple and generic.

\subsection{\textbf{Painting Application}}

Our first application is a painting application program. In this application, the user will be able to paint by blowing into the tube and the harder the user blows, the thicker the paint will be. Changing the color of the paint is achieved by rotating the tube. The painted color will disappear after some time. We implement the strokes to be brush-like, as if the user is painting with a real brush.
\begin{figure}
  \centering
  \includegraphics[width=\linewidth]{./figs/tube3.png}
  \caption{A screenshot of our painting application, drawn by our tester.}
  \label{fig:painting}
\end{figure}

\subsection{\textbf{Balloon Popping Game}}

The next application is a game in which the user is required to pass through a set of levels by popping down balloons that randomly appear on the screen, as shown in Figure~\ref{fig:shooting-game}. This is done by moving the pointer to where the balloons are and blowing into the tube. The game consists of two levels:  in the first level, the pointer and the balloons' color are all black so that user only needs to point and shoot. This level is intended to familiarize the user with the basic concept of controlling the movement of our \tube to pop the balloons. In the second level, red, blue and green balloons are randomly generated therefore the user will have to rotate the tube to change the color of the pointer and match it with the color of the balloon in order to pop it.

\begin{figure}
  \centering
  \includegraphics[width=0.70\linewidth]{./figs/tubemaster.png}
  \caption{A screenshot of the beginning of each level of our balloon popping game.}
  \label{fig:shooting-game}
\end{figure}

